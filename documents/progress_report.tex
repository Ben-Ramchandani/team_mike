\documentclass[a4paper,10pt]{article}
%\documentclass[a4paper,10pt]{scrartcl}

\usepackage{graphicx}
\usepackage{multicol}
\usepackage{float}
\usepackage{listings}
\usepackage{color}
\usepackage{makeidx}
\usepackage[margin=1.2in]{geometry}
\definecolor{dkgreen}{rgb}{0,0.6,0}
\definecolor{gray}{rgb}{0.5,0.5,0.5}
\definecolor{mauve}{rgb}{0.58,0,0.82}

\setlength{\textheight}{710pt}
\setlength{\topmargin}{-60pt}
\setlength{\parindent}{0pt}

\lstset{frame=tb,
  language=Java,
  aboveskip=3mm,
  belowskip=3mm,
  showstringspaces=false,
  columns=flexible,
  basicstyle={\small\ttfamily},
  numbers=none,
  numberstyle=\tiny\color{gray},
  keywordstyle=\color{blue},
  commentstyle=\color{dkgreen},
  stringstyle=\color{mauve},
  breaklines=true,
  breakatwhitespace=true,
  tabsize=3
}
\usepackage[utf8]{inputenc}
\usepackage{listings}
\title{}
\author{}
\date{}
\makeindex
\pdfinfo{%
  /Title    (Project Specifications)
  /Author   (Team Mike)
  /Creator  (Razvan)
  /Producer ()
  /Subject  (Requirements and Specifications)
  /Keywords ()
}

\begin{document}

\begin{titlepage}
	\centering
	
	{\scshape\Large Team Mike\par}
	\vspace{4cm}
	{\huge\bfseries Put your Phone to Work\par}
	\vspace{1.5cm}
	{\Large\
	Progress Report
	\par}
	\vspace{2cm}
	{\Large\itshape 
	      Ben Ramchadani\\
	      Dmitrij Szamozvancev\\
	      Razvan Kusztos\\
	      James Wood \\
	      Laura Nechita \\
	      Jack Needham
	      \par}
	\vfill

% Bottom of the page
	{\large \today\par}
\end{titlepage}
\maketitle
\tableofcontents
\newpage
\section{Aims and Goals}
\section{Server}

The server is working with the prime example, fulfilling the Must portion of the requirements.
Currently being hosted on an Amazon Web Services EC2 instance the server consists of 2356 lines of Java code, including tests,
as well as configuration and HTML files.


\subsection{The Play! Framework}
We have decided to use the Play! Framework. It is an open source server framework which is similar to Django for Python or Ruby on Rails. This enforced some constraints on our model, but also optimized the time taken to write the application. 
All the communication with the devices is done via HTTP requests at this point. Play! redirects them to a threaded controller pool and runs the appropriate method.


\subsection{PostgreSQL and data storage}
All computation state is kept in a database. We used PostgreSQL due to the advantages it provides: security, caching etc. However, manipulating the database is done at an abstract level via the Ebean persistence library. Using this had an effect on our previous UML diagram. All the building blocks of the server: Computations, Jobs and Data are now stored in the database. In order to make this possible, we had confine ourselves to the Ebean model, which involves simple fields and no inheritance. 


\subsection{Prime Computation Example}

A simple example computation, it takes a single argument, a number $N$, and either finds that the number is prime or returns an example factor for that number.
This example was chosen because it is easy to understand and easy to split up into independent jobs.

The method for finding a factor is very simplistic, checking every number from 2 to $N-1$. The jobs each have a range to check, split so there are about ten jobs per computation.
Once the jobs have been completed the server checks through them and checks if any of them found a factor.


\subsection{Process of fetching a job}

\subsubsection{Key}
\begin{itemize}
\item[+] Implemented
\item[$\star$] Partially implemented / implemented for PrimeComputation only.
\item[-] Not implemented
\end{itemize}

\subsubsection{Process}

\begin{itemize}
\item[$\star$] A customer adds a computation to the system via the Web interface.



\item[+] All the jobs necessary to complete the computation are generated and committed to the database.

A computation generator is found based on the name of the computation and called.
All the jobs necessary to complete the computation are explicitly generated when the computation is initialized.
While this could limit efficiency in some cases, it reduces the state held in the computation.

\item A device:
\begin{itemize}
\item[$\star$] Notifies the server it is available, sending information about itself.

A cookie is sent with the response as a way of identifying the device.
This cookie must be present in subsequent requests.\\
This has been implemented, however the extra information sent is currently ignored.

\item[+] Requests a job from the server.

Information about the job is sent as JSON in the response, including the job ID and the name of the code file to be sent.
Requires a session cookie to be present.

\item[+] Fetches the code for that job from the server.

The code is fetched using its full file name, so Java's built in URLClassLoader can be used.

\item[$\star$] Fetches Data for that job from the server.

This requires that the device has been given this job.

\item[+] Executes the job.


\item[$\star$] Returns the result to the server.

This has been implemented, however larger data sizes have not been tested.
\end{itemize}

\item[+] The server stores the results of jobs as they come in.

\item[+] Once a computation finishes the server runs code to collate the jobs.

\item[-] The result is made available via the Web interface.
\end{itemize}

\subsection{Unit Testing}

The Job server has unit tests covering most components as well as a test that covers the a full computation execution.
The JUnit framework is used along with Play!'s test helpers, allowing the tests to be run every time the server is built.\\
The unit tests run inside fake application wrappers provided by Play!, which allow the use of an in memory database during tests.
The full test starts the server and pretends to be a device by making HTTP requests to localhost,
the internal server state is only accessed directly to check the result is correct at the end.
The test runs a prime computation (on 4, which generates only one job) from start to finish.\\

We are confident enough in these tests to say that the server is ready for integration.

\subsection{Server side UI}
We have started building a UI for our whole application. This involves a quick About section, a section with a link for app downloading and form for performing the computation test. This should be a taster and shall produce statistics of the computation, such as the amount of time taken, the number of devices that was involved in this, number of failures etc. 

The UI is built using a free Bootstrap theme with various modifications to suit our own theme. The communication to the server is done either via jQuery, as it is the case of the Prime example, or using Scala and the helper functions available in the framework which provides easier file uploading methods. 

Next we plan on making a more general form for Customers where they can generate computations by providing Java functions and inputs to run on, either files or raw text(already supported).

\subsection{Refactoring Ideas}

\begin{itemize}
\item Change the protocol to HTTPS, which is mostly handled by the Play! Framework.
\item Refector the Data class to allow efficient storage of larger data sizes.
\end{itemize}

\subsection{Upcoming features}

\begin{itemize}
\item Allow the UI to see the results of computations in realtime.
\item Store data about customers and users  in the database and make them available to the relevant parties.
\item Make the types of computations available to the app and the web UI.
\end{itemize}

\subsection{Deployment}
We have used the free Amazon Web Services to deploy our server. It runs in a Ubuntu environment and we can control the runtime of the application via SSH.
After integration, we plan on acquiring a domain name.




\section{Android Twork App}


We decided to keep the original layout and concentrate on adding functionality to the existing buttons and menus. The new components were added in the order shown below:

\subsection{Local database} The database residing on the user's device, containing data about the running computations and jobs. Data between the UI and background service is passed through the database to ensure asynchronicity and persistence. The implementation uses SQLite, a popular SQL framework for Android. A helper class provides a clean interface for communicating with the database.
\subsection{Computation service} Actual computation in the app runs in a separate service, in a new thread. Android services provide persistent computation in the background which can run even if the application is not in view. Technically it is a foreground service, which ensures that the Android system does not kill the service prematurely -- in exchange, a notification must be displayed while the app is running. The service methods can be implemented in a single class and accessed via the UI, so the user can have some control over how and when computation is happening.
The basic code for the service has been written and tested against the server on a computer, but has not been integrated into the app yet.
\subsection{Facebook integration} The social features are one of our main incentives and reward systems in the app, so we needed seamless integration with social networks. The first step was Facebook integration, done through the Facebook API -- for now it's only possible to log in, sharing features can be added when there is something worth sharing.
\subsection{Achievements} The in-app reward system comes in the form of achievements, accessible from a separate menu. We plan to have some generic, built-in achievements (like "You performed 20 hours of computation"), as well as let the customers provide their own, computation-specific achievements (e.g. "You have found 5 primes").
\subsection{Setup}When the user launches the app for the first time, the device needs to connect to the server, set up the database, fetch computations and jobs and so on. To make this process explicit, a short, one-time setup activity was added where the user can select which computations they want to run and set a few preferences. Even though this process is short, implementing it took quite a long time, but it provided a better view into the workings of the UI system and a few of the new classes have already been reused in different components of the application.
\subsection{Statistics}Since we need statistics to give out achievements, we thought that the user might be interested to see them real-time, in a more visual way. We are using the MPAndroidChart library to create simple graphs and charts to show the user's progress.


\subsection{Plans for new features}

The major components of the application are now implemented, but still they have to be fully integrated and tested. Other features left to build are:

\begin{itemize}
    \item Managing battery, mobile data and location services
    \item Adding settings
    \item Social sharing
    \item Managing achievements
    \item Visuals
\end{itemize}


\end{document}
