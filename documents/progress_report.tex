\documentclass[a4paper,10pt]{article}
%\documentclass[a4paper,10pt]{scrartcl}

\usepackage{graphicx}
\usepackage{multicol}
\usepackage{float}
\usepackage{listings}
\usepackage{color}
\usepackage{makeidx}
\usepackage[margin=1.2in]{geometry}
\definecolor{dkgreen}{rgb}{0,0.6,0}
\definecolor{gray}{rgb}{0.5,0.5,0.5}
\definecolor{mauve}{rgb}{0.58,0,0.82}

\setlength{\textheight}{710pt}
\setlength{\topmargin}{-60pt}
\setlength{\parindent}{0pt}

\lstset{frame=tb,
  language=Java,
  aboveskip=3mm,
  belowskip=3mm,
  showstringspaces=false,
  columns=flexible,
  basicstyle={\small\ttfamily},
  numbers=none,
  numberstyle=\tiny\color{gray},
  keywordstyle=\color{blue},
  commentstyle=\color{dkgreen},
  stringstyle=\color{mauve},
  breaklines=true,
  breakatwhitespace=true,
  tabsize=3
}
\usepackage[utf8]{inputenc}
\usepackage{listings}
\title{}
\author{}
\date{}
\makeindex
\pdfinfo{%
  /Title    (Project Specifications)
  /Author   (Team Mike)
  /Creator  (Razvan)
  /Producer ()
  /Subject  (Requirements and Specifications)
  /Keywords ()
}

\begin{document}

\begin{titlepage}
	\centering
	
	{\scshape\Large Team Mike\par}
	\vspace{4cm}
	{\huge\bfseries Put your Phone to Work\par}
	\vspace{1.5cm}
	{\Large\
	Progress Report
	\par}
	\vspace{2cm}
	{\Large\itshape 
	      Ben Ramchadani\\
	      Dmitrij Szamozvancev\\
	      Razvan Kusztos\\
	      James Wood \\
	      Laura Nechita \\
	      Jack Needham
	      \par}
	\vfill

% Bottom of the page
	{\large \today\par}
\end{titlepage}
\maketitle
\tableofcontents
\newpage
\section{Aims and Goals}
\section{Server}

%Working now, with tests.


\subsection{The Computation model}

%We have different types of computations (Take files or just strings), only basic one implemented
% 2356 lines of Java (Just what we've written, includes tests)

\subsection{The Play! Framework}

\subsection{PostgreSQL and data storage}

\subsection{Prime Computation Example}

A simple example computation, it takes a single argument, a number $N$, and either finds that the number is prime or returns an example factor for that number.
This example was chosen because it is easy to understand and easy to split up into independent jobs.

The method for finding a factor is very simplistic, checking every number from 2 to $N-1$. The jobs each have a range to check, split so there are about ten jobs per computation.
Once the jobs have been completed the server checks through them and checks if any of them found a factor.

\subsection{What happens $<$-- terrible section title}

\subsubsection{Key}
\begin{itemize}
\item[+] Implemented
\item[$\star$] Partially implemented / implemented for PrimeComputation only.
\item[-] Not implemented
\end{itemize}

\subsubsection{Process}

\begin{itemize}
\item[$\star$] A customer adds a computation to the system via the Web interface.

\item[+] All the jobs necessary to complete the computation are generated and committed to the data base

\item A device:
\begin{itemize}
\item Notifies the server it is available, sending information about itself.
\item[+] Requests a job from the server.
\item[+] Fetches the code for that job from the server.
\item[$\star$] Fetches Data for that job from the server.
\item[+] Runs the job.
\item[$\star$] Returns the result to the server.
\end{itemize}

\item[+] The server stores the results of jobs as they come in.

\item[+] Once a computation finishes the server runs code to collate the jobs.

\item[-] The result is made available to the Web interface.
\end{itemize}

\subsection{Unit Testing}

The Job server has unit tests covering most components as well as a test that covers the a full computation execution.
The JUnit framework is used along with Play!'s test helpers, allowing the tests to be run every time the server is built.\\
The unit tests run inside fake application wrappers provided by Play!, which allow the use of an in memory database during tests.
The full test starts the server and pretends to be a device by making HTTP requests to localhost,
the internal server state is only accessed directly to check the result is correct at the end.
The test runs a prime computation (on 4, which generates only one job) from start to finish.\\

We are confident enough in these tests to say that the server is ready for integration.

\subsection{UI - incipient phase :)}

\subsection{Refactoring Ideas}

\begin{itemize}
\item Get rid of the Data
\item Allow the UI to see the result in realtime.
\end{itemize}

\section{Service}

\section{Android Twork App}

\end{document}
