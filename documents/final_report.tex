\documentclass[a4paper,10pt]{article}
%\documentclass[a4paper,10pt]{scrartcl}

\usepackage{graphicx}
\usepackage{multicol}
\usepackage{float}
\usepackage{listings}
\usepackage{color}
\usepackage{makeidx}
\usepackage[margin=1.2in]{geometry}
\definecolor{dkgreen}{rgb}{0,0.6,0}
\definecolor{gray}{rgb}{0.5,0.5,0.5}
\definecolor{mauve}{rgb}{0.58,0,0.82}

\setlength{\textheight}{710pt}
\setlength{\topmargin}{-60pt}
\setlength{\parindent}{0pt}

\lstset{frame=tb,
  language=Java,
  aboveskip=3mm,
  belowskip=3mm,
  showstringspaces=false,
  columns=flexible,
  basicstyle={\small\ttfamily},
  numbers=none,
  numberstyle=\tiny\color{gray},
  keywordstyle=\color{blue},
  commentstyle=\color{dkgreen},
  stringstyle=\color{mauve},
  breaklines=true,
  breakatwhitespace=true,
  tabsize=3
}
\usepackage[utf8]{inputenc}
\usepackage{listings}
\title{}
\author{}
\date{}
\makeindex
\pdfinfo{%
  /Title    (Project Report)
  /Author   (Team Mike)
  /Creator  (Razvan)
  /Producer ()
  /Subject  (Project Report)
  /Keywords ()
}

\begin{document}

\begin{titlepage}
	\centering
	
	{\scshape\Large Team Mike\par}
	\vspace{4cm}
	{\huge\bfseries Put your Phone to Work\par}
	\vspace{1.5cm}
	{\Large\
	Project Report
	\par}
	\vspace{2cm}
	{\Large\itshape 
	      Ben Ramchadani\\
	      Dmitrij Szamozvancev\\
	      Razvan Kusztos\\
	      James Wood \\
	      Laura Nechita \\
	      Jack Needham
	      \par}
	\vfill

% Bottom of the page
	{\large \today\par}
\end{titlepage}
\maketitle
\tableofcontents
\newpage
\section{Summary}

We have built a platform for performing distributed computation on idle mobile devices.

On top of this we have built examples for the purpose of the demo.
Via the web interface users may submit computations in the form of either prime number checking or one of several image manipulation tasks.
These tasks are then split up into jobs which are sent out to run on multiple devices.
Once the results have been collated, or even as they are coming in, they are returned in real time to the website.

The system is split conceptually into three parts: the server that manages the jobs and hosts the website, the service running on the device that does the computation
and the interface for the app that allows the end user to view information about and control the service.
All the parts are built to allow any form of computation, we have built examples intended for the demo.


\subsection{Server}

The server is written almost entirely in Java using the Play! framework and the the Ebeans persistence plug-in.

\subsubsection{Job management}
%this is a bit brief and not that well written
The server accepts new computations, splits them up into jobs each with their own bit of data, it manages handing out the jobs to the devices along with
the data and code needed to complete them, it receives the results of completed jobs, handles jobs and hence computations failing, as well
as data being returned when jobs or computations complete.
Information about jobs and computations is stored in the database and can be recovered
in case of a crash or restart.
Information about devices using the system is also kept persistently, including the number of jobs they have completed and failed.

\subsubsection{Website}


The website is designed to demo the system, though it could be expanded to allow for paying customers if Twork were to be made into a full product.
The website allows computations to be introduced into the system and results are returned in real time using WebSockets. It allows uploading a number of files and applying a function on them, being in tune with the server. New features can be added easily as soon as they become available. Some ideas for this are relevant statistics on computation completion, new types of functions supported etc.


\subsection{Twork app}

\section{Successes}

%It worksish
%It is very modular so we can easily extend it: we have the map file computation which can basically do anything given we have %the function, which is pretty cool
The project in general has been a success, however there are a few things we think went particularly well.

\subsection{Extensibility}

The system is very general in terms of the types of computation it can handle.
The code for computations interacts using Input and Output streams, the rest of the code are not affected by what they contain.
The strength of the abstractions used is demonstrated by the fact that image manipulation and prime testing are both managed by the same code.
The code for managing the image code on the server is implemented in terms of converting a set of files, any Java code that converts one file to another can be passed as a parameter.

%\subsection{Website}
% Since we weren't sure we'd have time to do this at all


%Need more things here

\section{Difficulties}

\subsection{Server}

The main challenge was leaning to use a completely new framework and persistence model, both of which have limited documentation.
We had some technical issues with the framework.\\

Ebean only partially loads classes from the database as an efficiency measure, leaving some fields as \texttt{null}.
Using getters and setters instead of direct access, using boxed instead of primitive types (\texttt{int} vs. \texttt{Integer}) and using different variable names
can all change what is loaded when. The on-line documentation mentions the partial loading, but does not explain it or how full loading can be forced in general.
We have fixed the problems, but they were hard to debug.\\

Play! also has some unexpected behaviour regarding (HTTP) request bodies, again with little documentation, getting the
request body as raw bytes may return \texttt{null} depending on the \texttt{content-type} and length.
We're using a solution we found on someone's personal blog who had the same problem.\\

Setting up the server, for example configuring the database and the EC2 instance took a long time
as neither of us had much experience.

\subsection{App interface}

\subsection{App service}
%Eg. class loading in Android
% problems with Play! docs
% problems with deploying on amazon web services
% security issues



%These sections can be removed if we don't want to write them
\section{Improvements}
\section{Acceptance criteria}

%Did we follow the spec?

\section{Summary of the work undertaken}

\subsection{Server}
%ben

%razvan

Razvan began by researching the various options for building the server, and he suggested that we should use the Play! Framework, as well as various other dependencies, such as the Ebean persistancy. Razvan then created the main data types for the original prototype, as well as the controllers for HTTP requests. 
Razvan then created a front end for our server, where people can add their own computations (currently supporting only simple image processing), as well as testing our framework with a simple  prime number checker. Results come back in real time.
%Who did what
\end{document}
