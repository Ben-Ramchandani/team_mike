\documentclass[a4paper,10pt]{article}
%\documentclass[a4paper,10pt]{scrartcl}

\usepackage{graphicx}
\usepackage{multicol}
\usepackage{float}
\usepackage{listings}
\usepackage{color}
\usepackage{makeidx}
\usepackage[margin=1.2in]{geometry}
\definecolor{dkgreen}{rgb}{0,0.6,0}
\definecolor{gray}{rgb}{0.5,0.5,0.5}
\definecolor{mauve}{rgb}{0.58,0,0.82}

\setlength{\textheight}{710pt}
\setlength{\topmargin}{-60pt}
\setlength{\parindent}{0pt}

\lstset{frame=tb,
  language=Java,
  aboveskip=3mm,
  belowskip=3mm,
  showstringspaces=false,
  columns=flexible,
  basicstyle={\small\ttfamily},
  numbers=none,
  numberstyle=\tiny\color{gray},
  keywordstyle=\color{blue},
  commentstyle=\color{dkgreen},
  stringstyle=\color{mauve},
  breaklines=true,
  breakatwhitespace=true,
  tabsize=3
}
\usepackage[utf8]{inputenc}
\usepackage{listings}
\title{}
\author{}
\date{}
\makeindex
\pdfinfo{%
  /Title    (Project Report)
  /Author   (Team Mike)
  /Creator  (Razvan)
  /Producer ()
  /Subject  (Project Report)
  /Keywords ()
}

\begin{document}

\begin{titlepage}
	\centering
	
	{\scshape\Large Team Mike\par}
	\vspace{4cm}
	{\huge\bfseries Put your Phone to Work\par}
	\vspace{1.5cm}
	{\Large\
	Project Report
	\par}
	\vspace{2cm}
	{\Large\itshape 
	      Ben Ramchadani\\
	      Dmitrij Szamozvancev\\
	      Razvan Kusztos\\
	      James Wood \\
	      Laura Nechita \\
	      Jack Needham
	      \par}
	\vfill

% Bottom of the page
	{\large \today\par}
\end{titlepage}
\maketitle
\tableofcontents
\newpage
\section{Summary}

%What it does

\section{Successes}

%It worksish
%It is very modular so we can easily extend it: we have the map file computation which can basically do anything given we have %the function, which is pretty cool

\section{Difficulties}

%Eg. class loading in Android
% problems with Play! docs
% problems with deploying on amazon web services
% security issues

\section{Acceptance criteria}

%Did we follow the spec?

\section{Summary of the work undertaken}

\subsection{Server}
%ben

%razvan

Razvan began by researching the various options for building the server, and he suggested that we should use the Play! Framework, as well as various other dependencies, such as the Ebean persistancy. Razvan then created the main data types for the original prototype, as well as the controllers for HTTP requests. 
Razvan then created a front end for our server, where people can add their own computations (currently supporting only simple image processing), as well as testing our framework with a simple  prime number checker. Results come back in real time.

\subsection{Service}

To start with, Jack and James prototyped parts of the service process that would be required, focussing on communication with the server and the interface, respectively. Ben ended up writing some code to communicate with the server, and this became the basis of the actual implementation.

After this, there was significant effort put into making the code more fault-tolerant and general. James wrote some code to handle loading of classes at runtime, including security provisions, but it was decided that this wouldn't be ready on the server side and tested in time for the demonstration.

%TODO: maybe a little more here. I'll do some later.

%Who did what
\end{document}
