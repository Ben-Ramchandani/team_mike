\documentclass[
  twoside,
  10pt, a4paper
]{article}


\usepackage{acronym}



\setlength{\parindent}{0pt}
\setlength{\oddsidemargin}{-20pt}
\setlength{\evensidemargin}{-20pt}
\setlength{\topmargin}{-60pt}
\setlength{\textwidth}{480pt}
\setlength{\textheight}{700pt}
\setlength{\marginparwidth}{0pt}

\title{Project Plan}
\author{Team Mike}

\begin{document}

\begin{titlepage}
	\centering
	
	{\scshape\Large Team Mike\par}
	\vspace{4cm}
	{\huge\bfseries Put your Phone to Work\par}
	\vspace{1.5cm}
	{\Large\
	Project plan
	\par}
	\vspace{2cm}
	{\Large\itshape 
	      Ben Ramchadani\\
	      Dmitrij Szamozvancev\\
	      Razvan Kusztos\\
	      James Wood \\
	      Laura Nechita \\
	      Jack Needham
	      \par}
	\vfill

% Bottom of the page
	{\large \today\par}
\end{titlepage}
\maketitle
\tableofcontents
\newpage

\section{Overview}

\subsection{Team members}

We have six team members:
\begin{itemize}
\item Ben Ramchadani
\item Dmitrij Szamozvancev
\item Razvan Kusztos
\item James Wood
\item Laura Nechita
\item Jack Needham
\end{itemize}



\subsection{Components and roles}

Ben Ramchandani is the project manager and point of contact.\\

The system is split into three main components:
\begin{itemize}
\item The job server and example computations.

Ben Ramchandani and Razvan Kusztos will work on the server.

\item The phone app user interface.

Dmitrij Szamozvancev and Laura Nechita will work on the user interface.

\item The phone app worker service.

James Wood and Jack Needham will work on the app service.
\end{itemize}



\newpage
\section{Implementation plan}


\subsection{Server}

Razvan Kusztos and Ben Ramchandani.

\subsubsection{Strategy}

Given that we have split the server into modules, we aim to follow an Agile development method, focusing on building a rapid horizontal prototype before fully implementing each module
and unit testing them. We hope this will allow us to get a minimum viable product within two weeks so we can begin implementing extra features.

\subsubsection{Prototype}

We aim to have a working prototype with very limited functionality ready in the next few days.
This will allows us to quickly find unforeseen problems with the design and give us a base for moving forward.\\

We then intend to fully implement the Network module, this allows us to test its design and also means that the AIP is fixed and available as soon as possible for the app to test against.\\

The other modules can be replaced with basic prototypes:\\
Computation module: A single example computation -- checking for prime numbers\\
Database module: Print requests to console\\
Scheduler: Just keep a list of jobs\\


\subsubsection{Implementation order}

Once the initial prototype is complete we will properly implement the rest of the modules to specification in this order.

\paragraph{Scheduler:} This allows us to test the system with multiple users and ensure it gives correct results.

\paragraph{Computation module:} This allows us to begin making and testing more example computations and templates.

\paragraph{Database module:} This can be left until now as in most cases the previous modules only need to read from, not write to, the database.

\paragraph{Computation Template Manager and Customer interface:} These are left until last as the system can be tested to a large extent with these replaced with hardcoded versions.


\subsubsection{Additional work}

After each part is implemented to the specification, if we have time we will start developing additional features.\\
For example:
\begin{itemize}
\item An interface for customer to use.
\item Our own HTTP server for giving out data, using a database back end.
\item Runtime loading of new computation templates.
\end{itemize}

\subsubsection{Unit tests}

We will split the modules between us when we implement them. We will write unit tests not only for our own modules, but for each others.
This ensures we have a common understanding of each module. The tests will use JUnit, the standard for Java.







\subsection{Device user interface}

Laura Nechita and Dmitrij Szamozvancev.

\subsubsection{Strategy} 
In the last two weeks the team members have familiarised themselves with Android development, with us paying particular attention to the user interface and Android-specific features and APIs. A basic mockup of the interface can be quickly built using the available tools, and the custom functionality can be integrated and tested in parallel with the other subteams.

\subsubsection{Prototype}
We plan to have a basic prototype ready in the next few days, hopefully before the client meeting. The app would already have the general layout and menu structure, but no functionality. The modules can be connected to the various UI elements as they are being developed, which allows for quick and continuous testing. 

\subsubsection{Implementation order}
With the basic prototype finished, we intend to add functionality to the components in roughly the following order:
\paragraph{Background service: } Set up and test the background execution service where the client code can run.
\paragraph{Android-specific features: } Functionality related to the Android platform (e.g. battery and data management, location services) which can be set up before the back-end service is fully implemented.
\paragraph{Database and social features: } Set up the database for the UI and service, add log-in options and basic social interactions.
\paragraph{Integration: } Build in the back-end components and make the system fully functional.

\subsubsection{Additional work} 
Additional, optional features that we will consider if the time permits it:
\begin{itemize}
    \item Visualisation of the statistics (graphs, charts, etc.)
    \item Additional settings and options for the user
    \item A promotional website for the application and the service
\end{itemize}

\subsubsection{Testing}
Testing of the user interface elements is made easy by the built-in emulator tool and the ability to use the application on an actual phone by connecting it to the program. Once the application is functional, it can be tested by running the example computations on the team members' phones, and we can even try to find a larger testing group to examine the scalability of our solution.


\newpage
\subsection{Device service}

James Wood and Jack Needham

\subsubsection{Strategy}

The background service will be built closely in tandem with the user interface portion of the app, so we will be communicating closely with them and maintaining feature parity.

\subsubsection{Prototype}

We intend to build the prototype in two stages, these will then be used together to guide the rest of the implementation.

A service that interacts with the user interface portion of the app will be built, so we can ensure data can be passed between the processes.
This will also ensure we have a functioning service that will persist in the background.

The next part is a prototype that interacts with the server using the specified API. We should test this works as soon as we can so we can fix any problems that could arise.

\subsubsection{Implementation}

Once the prototype is done we will implement the core functionality that runs the computation on the phone and ensure that is working with the other components before adding more features.

\subsubsection{Additional work}
Additional features that may be added once the core functionality is complete, for example:
\begin{itemize}
\item Notifying the app user interface of current jobs/computations
\item Functionality to fetch extra data for the computation over HTTP when requested
\item Being able to reduce the priority and suspend computations without losing progress
\item Sending additional data to the server, like location and remaining battery
\end{itemize}


\subsection{Integration testing}

Part of our strategy is to have prototype implementations for the interfaces between components as soon as possible, which should minimize problems when integrating.

\paragraph{Stress test}

Once we have a minimal viable product we will test the system with an image feature extraction computation. This will need to distribute images across the connected devices.
We use this example as it utilizes all the server modules and, since it is an intensive computation, we will be able to deliver some representative benchmarks.
These benchmarks will allow us to plan any tweaks to the system before the final deadline.

\end{document}
